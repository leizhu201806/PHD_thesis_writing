
\chapter{透过散射介质3D目标成像}

在前面章节中,我们对基于OME卷积成像模型进行了介绍,通过数值仿真和实验验证的方式,对散斑自相关的基本成像原理进行研究,研究了不同相位恢复算法的特点实现透过散射介质彩色成像和基于散斑的之间相的关性提出新型的非入侵透过散射介质成像方法。在第\ref{chap:5}章节中,我们所提出的方法能够有效的恢复超出光学记忆的目标,实现了对于扩展目标的非入侵成像。然后,如何有效的实现透过散射介质3D成像仍具有挑战性,以上的方法都具有各自的局限性,无法有效地解决透过散射介质的3D成像。

因此,本章中我们提出了一种多帧散斑照明的3D成像方法。与第\ref{chap:5}章节中所展示的结果不同,本章我们重点研究当空间存在多个目标,但不同的目标位于不同OME范围,如何进行有效地进行图像重建。受到第\ref{chap:5}章节方法的启发,我们可以通过随机照明的方式,获得系统中不同点光源的散斑指纹,但是当有效的获取散斑指纹后该如何进行重建将在接下来部分进行讨论。受到机器学习,图像分类和模式识别相关工作的启发,我们提出了一种无监督的散斑指纹分类方法,对散斑指纹进行分类,并最终实现了透过散射介质的3D成像。我们的方法特点在于:(\romannum{1})只需散斑照明;(\romannum{2})只需散斑照明

\section{公式}
