
\begin{abstract}
生活中,我们观察物体、辨别物体总是遵循“光沿直线传播”、“所见即所得”的传统光学成像规律,常见的视觉成像系统像人眼、照相机以及光学成像工具如放大镜、显微镜和望远镜等都以此为物理基础。传统光学成像主要通过提取弹道光(或抑制散射光)的方式解决透过无散射(或者弱散射)介质的成像问题。当光透过生物组织、烟尘和云雾等强散射介质时,传统光学成像规律失效。其主要原因是光在强散射介质中传播时,介质中大小为波长量级的粒子对入射光波产生的散射作用使原本有序的波前相位产生严重畸变,出射光场变得随机且紊乱,最终在观测面上只能接收到散斑图案,难以实现对目标的直接观测或成像。因此,散射效应成为制约透过散射介质成像技术发展的瓶颈和关键问题。随着工业水平和光学加工水平的飞速发展,空间光调制器和数字微镜器件的问世,使得对入射光场实现控制变得可能。此类方法,通常被称为波前调制技术,通过控制入射光场,实现透过散射介质聚焦或成像。波前调制技术通常需要比较繁琐的优化过程,而且通常难以通过非入侵的形式实现。由于波前整形技术在实际应用中有着极大限制,常常它成像需求难以匹配实际成像条件,因此在实际应用中受到极大的限制。与此同时,随着计算机技术飞速发展、介观物理研究的深入、计算成像思想的完善和图像处理技术的发展,形成以物理机制为基础的计算光学成像技术。计算光学成像技术作为新型成像手段,不仅推动了传统成像的发展,而且在解决散射成像方面表现出了得天独厚的优势。由于通过计算的方式实现图像重建,无需连续的波前优化等相关复杂过程,通常获得散斑数据后以”离线”的方式在计算机上完成计算重建。利用散斑信号,通过计算的方式实现图像重建,通常依赖于介质的自身属性,如:光学记忆效应;或者通过预标定的方式获取散射介质的物理特性。相较于波前调制方式,散射计算成像更具挑战性,但同时也拥有更广的应用前景。近年来,科研人员利用散射介质的光学记忆效应,实现透过散射非入侵散射成像。

。\\
中文摘要格式要求为:宋体小四、两端对齐、首行缩进~2~字符,行距为固定值~20~磅,段落间距为段前~0~磅,段后~0~磅。\\
英文摘要格式要求为:Times New Roman、小四、两端对齐、首行不缩进,行距为固定值~20~磅,段落间距为段前~0~磅,段后~0~磅,段与段之间空一行。\\

\keywords{XXX,\quad{}XXX,\quad{}XXX,\quad{}XXX,\quad{}XXX} \\
\end{abstract}

\begin{englishabstract}
The Abstract is a brief description of a thesis or dissertation without notes or comments. It represents concisely the research purpose, content, method, result and conclusion of the thesis or dissertation with emphasis on its innovative findings and perspectives. The Abstract Part consists of both the Chinese abstract and the English abstract. The Chinese abstract should have the length of approximately 1000 Chinese characters for a master thesis and 1500 for a Ph.D. dissertation. The English abstract should be consistent with the Chinese one in content. The keywords of a thesis or dissertation should be listed below the main body of the abstract, separated by commas and a space. The number of the keywords is typically 3 to 5.
\\~\\
The format of the Chinese Abstract is what follows: Song Ti, Small 4, justified, 2 characters indented in the first line, line spacing at a fixed value of 20 pounds, and paragraph spacing section at 0 pound.
\\~\\
The format of the English Abstract is what follows: Times New Roman, Small 4, justified, not indented in the first line, line spacing at a fixed value of 20 pounds, and paragraph spacing section at 0 pound with a blank line between paragraphs.
~\\
\englishkeywords{XXX,\space{}XXX,\space{}XXX,\space{}XXX,\space{}XXX} \\

\end{englishabstract}


\XDUpremainmatter

\begin{symbollist}
\item [符号] \hspace{12em} {符号名称}
\item [$\mu_{a}$]\hspace{12.5em} {吸收系数}
\item [$\mu_{s}$]\hspace{12.5em} {散射系数}
\item [$l_{s}$] \hspace{12.5em} {平均自由程}
\item [$\langle \cdot \rangle$] \hspace{12.5em} {均值}
\item [$\bigstar$] \hspace{12.5em} {相关}
\item [$*$] \hspace{12.5em} {卷积}
\item [$\delta$] \hspace{12.5em} {$\delta$函数}
\item [$\mathcal{F}$] \hspace{12.5em} {傅里叶变换}
\item [$\mid \cdot \mid $] \hspace{12.5em} {模值}
\item [$\lambda$] \hspace{12.5em} {波长}
\item [$\mu$] \hspace{12.5em} {标准差}
\item [$(\cdot)^{(3)}$] \hspace{12.5em} {三阶相关}
\item [$\Arrowvert \cdot \Arrowvert_F^{2}$] \hspace{12.5em} {Frobenius范数}
\item [$\vert\vert \mathbf{f}\vert\vert_2  $] \hspace{12.5em} {$L_{2}$向量范数}
\item [$\vert\vert \mathbf{f} \vert\vert_{TV} $] \hspace{12.5em} {TV范数}
\end{symbollist}

\begin{abbreviationlist}
\item \makebox[8em][l]{缩略语}  \makebox[16em][l]{英文全称}  \makebox[16em][l]{中文对照}
\item \makebox[4em][l]{OME}    \makebox[20em][l]{Optical Memory Effect}    \makebox[16em][l]{光学记忆效应}
\item \makebox[4em][l]{SLM}    \makebox[20em][l]{Spatial light modulator}    \makebox[16em][l]{空间光调制器}
\item \makebox[4em][l]{DMD}    \makebox[20em][l]{Digital micromirror device}  \makebox[16em][l]{数字微镜器件}
\item \makebox[4em][l]{2D}    \makebox[20em][l]{Two Dimensional}    \makebox[16em][l]{二维}
\item \makebox[4em][l]{3D}   \makebox[20em][l]{Three Dimensional}    \makebox[16em][l]{三维}
\item \makebox[4em][l]{MFP}    \makebox[20em][l]{Mean-Free Path}    \makebox[16em][l]{平均自由程}
\item \makebox[4em][l]{SSTM}    \makebox[20em][l]{Spatial-Spectral Transmission Matri}    \makebox[16em][l]{空间-光谱传输矩阵}
\item \makebox[4em][l]{HIO}    \makebox[20em][l]{Hybrid Input-Output}    \makebox[16em][l]{混合输入输出}
\item \makebox[4em][l]{ER}    \makebox[20em][l]{Error reduction}    \makebox[16em][l]{误差减小}
\item \makebox[4em][l]{TR}    \makebox[20em][l]{Tikhonov regularization}    \makebox[16em][l]{吉洪诺夫正则化算法}
\item \makebox[4em][l]{CVX}   \makebox[20em][l]{Convex Optimization}    \makebox[10em][l]{凸优化算法}
\item \makebox[4em][l]{FWHM}    \makebox[20em][l]{Full Width Half Maximum}    \makebox[16em][l]{全宽半高}
\item \makebox[4em][l]{sPSF}    \makebox[20em][l]{Spectral Point Spread Function}    \makebox[16em][l]{光谱点扩散函数}
\item \makebox[4em][l]{OTF}    \makebox[20em][l]{Optical Transfer Function}    \makebox[16em][l]{光学传递函数}
\item \makebox[4em][l]{TV}    \makebox[20em][l]{Total variation}    \makebox[16em][l]{总变差}
\item \makebox[4em][l]{ALS}    \makebox[20em][l]{Alternating Least Squares}    \makebox[16em][l]{交替最小二乘法}
\item \makebox[4em][l]{MDS}    \makebox[20em][l]{Multidimensional scaling}    \makebox[16em][l]{多维缩放}
\end{abbreviationlist}
