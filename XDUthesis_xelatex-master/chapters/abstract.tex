
\begin{abstract}
生活中,我们观察物体、辨别物体总是遵循“光沿直线传播”、“所见即所得”的传统光学成像规律,常见的视觉成像系统像人眼、照相机以及光学成像工具如放大镜、显微镜和望远镜等都以此为物理基础。传统光学成像主要通过提取弹道光(或抑制散射光)的方式解决透过无散射(或者弱散射)介质的成像问题。当光透过生物组织、烟尘和云雾等强散射介质时,传统光学成像规律失效。其主要原因是光在强散射介质中传播时,介质中大小为波长量级的粒子对入射光波产生的散射作用使原本有序的波前相位产生严重畸变,出射光场变得随机且紊乱,最终在观测面上只能接收到散斑图案,难以利用弹道光实现成像。因此,散射效应成为制约透过散射介质成像技术发展的瓶颈和关键问题。

随着工业水平和光学加工水平的飞速发展,空间光调制器和数字微镜器件的问世,使得对入射光场实现控制变得可能。此类方法,通常被称为波前调制技术,通过控制入射光场,实现透过散射介质聚焦或成像。波前调制技术通常需要比较繁琐的优化过程,而且通常难以通过非入侵的形式实现,因此在实际应用中受到极大的限制。与此同时,随着计算机技术飞速发展、介观物理研究的深入、计算成像思想的完善和图像处理技术的发展,形成以物理机制为基础的计算光学成像技术。计算光学成像技术作为新型成像手段,不仅推动了传统成像的发展,而且在解决散射成像方面表现出了得天独厚的优势。由于通过计算的方式实现图像重建,无需连续的波前优化等相关复杂过程,通常获得散斑数据后以“离线”的方式在计算机上完成计算重建。利用散斑信号,通过计算的方式实现图像重建,通常依赖于介质的自身属性,如:光学记忆效应;或者通过预标定的方式获取散射介质的物理特性。相较于波前调制方式,散射计算成像更具挑战性,但同时也拥有更广的应用前景。近年来,科研人员利用散射介质的光学记忆效应,实现透过散射非入侵散射成像。虽然基于光学效应的散斑相关成像等方法实现了非入侵的透过散射介质成像,但是依然有三方面问题亟待解决:(\romannum{1})现有的基于光学记忆效应的散斑自相关成像方法需要借助于相位恢复算法进行图像恢复,受到相位恢复机理限制,隐藏目标的方向常常被忽略,进而导致该方法难以进行透过散射介质彩色成像的扩展;(\romannum{2})现有的基于光学记忆效应的散斑相关成像方法的视场受到光学记忆效应的物理特性限制,难以实现大视场非入侵成像;(\romannum{3})散射介质的光学记忆效应范围随着深度的增加而减小,当透过散射介质在一定深度进行散射成像且系统拥有多目标时,多目标信息混叠和光学记忆效应范围的急剧减小给现有的散射成像带来了极大的挑战。对于现有非入侵散射成像方法存在的不足,本文主要做了以下针对性工作研究;

(1)首先从理论出发,分析并研究了基于光谱传输矩阵的光谱重建方法和散斑自相关成像技术,分别对它们进行了数字和实验验证;然后,分析了光谱带宽变化对于光谱重建和图像重建的影响;其次,基于光谱传输矩阵的基本原理,将其扩展至空间域,利用光谱信息实现散射成像;

(2)针对透过散射介质难以实现彩色成像的问题,提出了一种基于三阶相关相位恢复的彩色成像方法。该项共工作的核心在于:三阶相关相位恢复和振幅恢复步骤的相互独立性,实现了目标方向信息的确定性恢复。首次验证了无需波前整形,且以非侵入形式实现透过散射介质的彩色成像。相较于传统的透过散射介质彩色成像方法,该方法无需对于目标的先验知识,无需介质波前整形技术,在时效性方面更具优势。同时,该方法能够与传统的光谱成像方法进行有效的结合,实现透过散射介质的光谱成像;

(3)针对散射成像视场首先问题,提出了一种基于波动随机照明的透过散射介质超光学记忆效应范围成像方法。目前已有的透过散射介质成像方法,难以克服光学视场首先问题,并且在受限的视场下,图像重建质量受到相位恢复算法的影响,难以确定的且高质量的完成图像重建。通过研究光学记忆效应范围的内在联系,利用多帧随机照明的方式获得不同的散斑图像,并利用去混叠算法实现不同点源目标的散斑指纹恢复,利用散斑指纹之间的相关特性,实现了透过散射介质超光学记忆效应范围成像。在此工作中,我们首先提出了基于散斑指纹的图像重建概念。

(4)提出了一种基于多帧散斑照明的散射介质多光学记忆效应范围成像方法。针对散射成像多目标混叠问题,研究了单一光学记忆效应范围与多光学记忆效应范围的差异,挖掘散斑包含的内在信息,通过散斑分类的方式解决了透过散射介质多目标成像困难的问题。同时,该方法的应用范围不局限于空间光学记忆效应,对于时间维光学记忆效应和光谱维光学记忆效应同样适用。

\keywords{散射成像,\quad{}散斑,\quad{}光绪记忆效应,\quad{}非入侵成像,\quad{}散射介质} \\
\end{abstract}

\begin{englishabstract}
In daily life, we observe and distinguish objects that always following the traditional optical imaging laws of "light travels in a straight line" and "what you see is what you get". Common visual imaging systems such as the human eye, cameras, and optical imaging tools such as magnifying glasses, microscopes, telescopes, etc. are based on this physical basis. Traditional optical imaging mainly solves the imaging problem through non-scattering (or weakly scattering) media by extracting ballistic light (or suppressing scattered light). When light passes through strong scattering media such as biological tissue, smoke, and clouds, the traditional optical imaging laws fail. The main reason is that when light propagates in a strongly scattering medium, the scattering effect of the wavelength-ordered particles in the medium on the incident light wave seriously distorts the originally ordered wavefront phase, and the outgoing light field becomes random and disordered. In the end, only speckle patterns can be received on the observation surface, and it is difficult to use ballistic light to achieve imaging. Therefore, the scattering effect has become the bottleneck and key problem restricting the development of imaging technology through a scattering medium.

With the rapid development of industrial level and optical processing level, the advent of spatial light modulators and digital micromirror devices makes it possible to control the incident light field. Such methods often referred to as wavefront modulation techniques enable focusing or imaging through a scattering medium by controlling the incident light field. Wavefront shaping technology usually requires a tedious optimization process, and it is usually difficult to achieve in a non-invasive form, so it is greatly limited in practical applications. At the same time, with the rapid development of computer technology, the deepening of mesoscopic physics research, the improvement of computational imaging ideas, and the development of image processing technology, computational optical imaging technology based on physical mechanisms have been formed. As a new imaging method, computational optical imaging technology not only promotes the development of traditional imaging but also shows unique advantages in solving scattering imaging. Since the image reconstruction is realized by calculation, there is no need for continuous wavefront optimization and other related complex processes, and the calculation and reconstruction are usually completed on the computer in an "offline" manner after speckle data is obtained. Using the speckle signal, the image reconstruction is realized by calculation, which usually depends on the properties of the medium, such as optical memory effect; or the physical properties of the scattering medium are obtained by pre-calibration. Compared with wavefront modulation, scattering computational imaging is more challenging, but it also has broader application prospects. In recent years, researchers have used the optical memory effect of scattering media to achieve non-invasive scattering imaging through scattering. Although methods such as speckle correlation imaging based on optical effects have achieved non-invasive imaging through scattering media, there are still three problems to be solved: (\romannum{1}) The existing speckle autocorrelation imaging based on optical memory effect The method needs to use a phase recovery algorithm for image recovery. Due to the limitation of the phase recovery mechanism, the direction of the hidden target is often ignored, which makes it difficult for the method to extend color imaging through scattering media; (\romannum{2}) The existing field of view of the speckle correlation imaging method based on the optical memory effect is limited by the physical characteristics of the optical memory effect, and it is difficult to achieve non-invasive imaging with a large field of view; (\romannum{3}) The range of the optical memory effect of scattering media increases with depth, However, when the scattering imaging is performed at a certain depth through the scattering medium and the system has multiple targets, the multi-target information aliasing and the sharp reduction of the optical memory effect range bring great challenges to the existing scattering imaging. For the shortcomings of the existing non-invasive scattering imaging methods, this paper mainly does the following research:

(1) Firstly, the spectral reconstruction method and speckle autocorrelation imaging technology based on the spectral transfer matrix are analyzed and studied from the theoretical point of view, and they are verified numerically and experimentally respectively; The influence of reconstruction; secondly, based on the basic principle of the spectral transmission matrix, it is extended to the spatial domain, and the spectral information is used to realize scattering imaging;

(2) Aiming at the problem that color imaging is difficult to achieve through scattering media, a color imaging method based on third-order correlation phase recovery is proposed. The core of this joint work lies in the mutual independence of the third-order correlation phase recovery and amplitude recovery steps, which realizes the deterministic recovery of the target direction information. For the first time, color imaging through scattering media has been demonstrated in a non-invasive manner without wavefront shaping. Compared with the traditional color imaging method through scattering medium, this method does not require prior knowledge of the target and does not require medium wavefront shaping technology, and has more advantages in terms of timeliness. At the same time, this method can be effectively combined with traditional spectral imaging methods to achieve spectral imaging through scattering media;

(3) Aiming at the first problem of scattering imaging field of view, a method for imaging through scattering medium beyond the optical memory effect range based on fluctuating random illumination is proposed. The existing imaging methods through scattering media are difficult to overcome the first problem of the optical field of view, and in the limited field of view, the image reconstruction quality is affected by the phase recovery algorithm, and it is difficult to complete the image reconstruction with certainty and high quality. By studying the intrinsic relationship of the optical memory effect range, different speckle images are obtained by using multi-frame random illumination, and the anti-aliasing algorithm is used to recover the speckle fingerprints of different point source targets, and the correlation characteristics between the speckle fingerprints are used. , to achieve imaging through the scattering medium beyond the optical memory effect range. In this work, we first propose the concept of image reconstruction based on speckle fingerprints.

(4) A multi-optical memory effect ranges imaging method for scattering media based on multi-frame speckle illumination is proposed. Aiming at the multi-target aliasing problem in scattering imaging, the difference between the single optical memory effect range and the multi-optical memory effect range is studied, and the inherent information contained in the speckle is mined. problem. At the same time, the application range of this method is not limited to the spatial optical memory effect, but also applies to the time-dimensional optical memory effect and the spectral-dimensional optical memory effect.\\
\englishkeywords{scattering imgaging,\space{}speckle pattern,\space{}optical memory effect,\space{}non-invasive imaging,\space{}scattering media} \\

\end{englishabstract}


\XDUpremainmatter

\begin{symbollist}
\item [符号] \hspace{12em} {符号名称}
\item [$\mu_{a}$]\hspace{12.5em} {吸收系数}
\item [$\mu_{s}$]\hspace{12.5em} {散射系数}
\item [$l_{s}$] \hspace{12.5em} {平均自由程}
\item [$\langle \cdot \rangle$] \hspace{12.5em} {均值}
\item [$\bigstar$] \hspace{12.5em} {相关}
\item [$*$] \hspace{12.5em} {卷积}
\item [$\delta$] \hspace{12.5em} {$\delta$函数}
\item [$\mathcal{F}$] \hspace{12.5em} {傅里叶变换}
\item [$\mid \cdot \mid $] \hspace{12.5em} {模值}
\item [$\lambda$] \hspace{12.5em} {波长}
\item [$\mu$] \hspace{12.5em} {标准差}
\item [$(\cdot)^{(3)}$] \hspace{12.5em} {三阶相关}
\item [$\Arrowvert \cdot \Arrowvert_F^{2}$] \hspace{12.5em} {Frobenius范数}
\item [$\vert\vert \mathbf{f}\vert\vert_2  $] \hspace{12.5em} {$L_{2}$向量范数}
\item [$\vert\vert \mathbf{f} \vert\vert_{TV} $] \hspace{12.5em} {TV范数}
\end{symbollist}

\begin{abbreviationlist}
\item \makebox[8em][l]{缩略语}  \makebox[16em][l]{英文全称}  \makebox[16em][l]{中文对照}
\item \makebox[4em][l]{OME}    \makebox[20em][l]{Optical Memory Effect}    \makebox[16em][l]{光学记忆效应}
\item \makebox[4em][l]{SLM}    \makebox[20em][l]{Spatial light modulator}    \makebox[16em][l]{空间光调制器}
\item \makebox[4em][l]{DMD}    \makebox[20em][l]{Digital micromirror device}  \makebox[16em][l]{数字微镜器件}
\item \makebox[4em][l]{2D}    \makebox[20em][l]{Two Dimensional}    \makebox[16em][l]{二维}
\item \makebox[4em][l]{3D}   \makebox[20em][l]{Three Dimensional}    \makebox[16em][l]{三维}
\item \makebox[4em][l]{MEMS}   \makebox[20em][l]{Micro electro mechanical system}    \makebox[16em][l]{微机电系统调制器}
\item \makebox[4em][l]{MFP}    \makebox[20em][l]{Mean-Free Path}    \makebox[16em][l]{平均自由程}
\item \makebox[4em][l]{SSTM}    \makebox[20em][l]{Spatial-Spectral Transmission Matri}    \makebox[16em][l]{空间-光谱传输矩阵}
\item \makebox[4em][l]{HIO}    \makebox[20em][l]{Hybrid Input-Output}    \makebox[16em][l]{混合输入输出}
\item \makebox[4em][l]{ER}    \makebox[20em][l]{Error reduction}    \makebox[16em][l]{误差减小}
\item \makebox[4em][l]{TR}    \makebox[20em][l]{Tikhonov regularization}    \makebox[16em][l]{吉洪诺夫正则化算法}
\item \makebox[4em][l]{CVX}   \makebox[20em][l]{Convex Optimization}    \makebox[10em][l]{凸优化算法}
\item \makebox[4em][l]{FWHM}    \makebox[20em][l]{Full Width Half Maximum}    \makebox[16em][l]{全宽半高}
\item \makebox[4em][l]{sPSF}    \makebox[20em][l]{Spectral Point Spread Function}    \makebox[16em][l]{光谱点扩散函数}
\item \makebox[4em][l]{OTF}    \makebox[20em][l]{Optical Transfer Function}    \makebox[16em][l]{光学传递函数}
\item \makebox[4em][l]{TV}    \makebox[20em][l]{Total variation}    \makebox[16em][l]{总变差}
\item \makebox[4em][l]{ALS}    \makebox[20em][l]{Alternating Least Squares}    \makebox[16em][l]{交替最小二乘法}
\item \makebox[4em][l]{SSIM}    \makebox[20em][l]{Structural Similarity Index Metric}    \makebox[16em][l]{结构相似性指数度量}
\item \makebox[4em][l]{MDS}    \makebox[20em][l]{Multidimensional scaling}    \makebox[16em][l]{多维缩放}
\end{abbreviationlist}
