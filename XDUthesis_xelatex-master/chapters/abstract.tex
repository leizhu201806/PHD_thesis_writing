\begin{abstract}

非侵入式光学成像在生物成像和光学检测等领域中都有重要的应用。但是,不均匀的样品(例如生物组织)会散射光,从而导致探测器上形成复杂的散斑图案。随着穿透深度的增加,从散射光中分离出少量的弹道光成为一个很大的挑战。目前,已经提出了几种技术通过使用反馈信号优化入射光波前进而实现聚焦,然后利用扫描的方式实现成像。这些技术通常需要途径至散射层的两侧获取反馈信号以优化波前,极大地限制了它们在实际场景中的应用。为了克服这个问题,提出了基于波前整形和各种反馈信号(例如荧光或超声信号)的其他方式,实现波前整形。然而,这些方法要么需要较长的采集时间,要么仅限于小视场,很难满足实用应用的需求,如生物医学成像等应用。

受到光学记忆效应的启发,科研人员提出了散斑相关成像技术,该技术实现了透过散射介质非入侵成像。随着研究的更加深入,验证了基于光学记忆效应的散斑相关成像方法同样具有较高的时间分辨率。虽然现有方法能够实现透过散射介质非入侵成像,但是散斑相关成像技术依然有三方面问题亟待解决:(\romannum{1})现有的基于光学记忆效应的散斑自相关成像方法需要借助于相位恢复算法进行图像恢复,受到相位恢复机理限制,隐藏目标的方向通常被忽略,进而导致该方法难以扩展至彩色成像;(\romannum{2})散斑相关成像方法的视场受到光学记忆效应的物理特性限制,难以实现大视场非入侵成像;(\romannum{3})散射介质的光学记忆效应范围随着深度的增加而减小,当透过散射介质在一定深度进行成像且系统拥有多目标时,多目标信息混叠和光学记忆效应范围的急剧减小给该方法带来了极大的挑战。针对现有非入侵散射成像方法存在的不足,本文主要做了以下针对性工作研究:

(1)针对目前已有散斑相关成像和光谱传输矩阵的光谱重建所存在的问题,分析并研究了光谱传输矩阵的光谱重建和散斑自相关成像的基本原理,分别对它们进行了数字和实验验证;并对光谱传输矩阵方法进行扩展,利用光谱传输方法实现散射成像;

(2)提出了一种基于三阶相关相位恢复的彩色成像方法。该项工作的核心在于:利用三阶相关相位恢复和振幅恢复步骤的相互独立性,实现了目标方向信息的确定性恢复和透过散射介质彩色成像。相较于传统的透过散射介质彩色成像方法,该方法无需对入射光波前进行连续优化,在时效性方面更具优势。同时,该方法能够与传统的光谱成像方法进行有效的结合,实现透过散射介质的光谱成像;

(3)提出了一种基于时变随机照明的透过散射介质的超光学记忆效应范围非入侵成像方法。该方法利用去混叠算法对多帧散斑进行去混叠,获得不同点光源的散斑指纹,然后通过成对去卷积的方式实现了透过散射介质的超光学记忆效应图像重建;

(4)提出了一种基于多帧散斑照明的散射介质多光学记忆效应范围成像方法。该方法首先通过去混叠算法对多帧散斑进行去混叠获取不同点光源的散斑指纹,然后利用散斑分类方式对散斑指纹进行分类,最后利用成对去卷积的方式对多光学记忆效应范围分别进行图像重建。该方法具有在时间维光学记忆效应和光谱维光学记忆应用的潜力。

\keywords{散射成像,\quad{}散斑,\quad{}光学记忆效应,\quad{}非入侵成像,\quad{}散射介质} \\
\end{abstract}

\begin{englishabstract}
Non-invasive optical imaging has important applications in various fields ranging from biotechnology to optical detection. However, inhomogeneous samples, such as biological tissues, scatter light, which results in a complex speckle pattern on the detector. With increasing depth, separating the low amount of ballistic light from the scattered light becomes a big challenge.  Several techniques have been proposed to focus light by making use of feedback signals to optimize the incident wavefront to recreate a focus that is then used for raster-scanning microscopy. These techniques require access to both sides of the scattering layer to optimize the wavefront, which strongly limits their application in real-case scenarios. To overcome this, other strategies have been proposed based on wavefront shaping and various feedback signals such as fluorescence or ultrasound signals. However, these approaches either require long acquisition times, entail the use of interferometric detection systems, or are limited to small fields of view (FoV). Therefore, those techniques are difficult to meet the requirements of practical applications, such as biomedical imaging applications.

Inspired by the optical memory effect, researchers have proposed a speckle correlation imaging technique, which enables non-invasive imaging through scattering media. Furthermore, it is verified that the speckle correlation imaging technique based on the optical memory effect also has a high temporal resolution. Although the current approaches can achieve non-invasive imaging through scattering media, they are suffering from the following challenging problems: (\romannum{1}) The existing speckle correlation imaging methods based on optical memory effect require retrieving the fourier phase information of the hidden object using a phase retrieval algorithm, the direction of the hidden target is often ignored, which makes it difficult to extend the method to color imaging; (\romannum{2}) The field of view of the speckle correlation imaging method is limited by the optical memory. Due to the limitation of the physical characteristics of the effect, it is difficult to achieve non-invasive imaging with a large field of view; (\romannum{3}) The optical memory effect range of the scattering medium decreases with the increase of depth. When imaging is performed at a certain depth through the scattering medium and the system When there are multiple targets, the multi-target information aliasing and the sharp reduction of the optical memory effect range bring great challenges to this method. Aiming at the shortcomings of the existing non-invasive scattering imaging methods, this paper mainly does the following research:

(1) Aiming at the current problems of speckle correlation imaging and spectral reconstruction of spectral transmission matrix, the basic principles of spectral reconstruction of spectral transfer matrix and speckle autocorrelation imaging are studied, and they are verified numerically and experimentally respectively. Further, we successfully reconstruct the image of the hidden object using the extended spectral transmission matrix technique.

(2) A color imaging through scattering media based on phase retrieval with triple correlation is proposed. The core of this joint work lies in the mutual independence of the triple correlation phase recovery and amplitude recovery steps, which realize the deterministic recovery of the target direction information. For the first time, color imaging through scattering media has been demonstrated in a non-invasive manner without a wavefront shaping technique. Compared with the traditional color imaging method through a scattering medium, this skill does not require correcting the wavefront of the incident beam and has more advantages in terms of timeliness. At the same time, this technique can be effectively combined with traditional spectral imaging approaches to achieve spectral imaging.

(3) A large field-of-view non-invasive imaging through scattering layers using fluctuating random illumination is put forward. The technique achieves this by demixing speckle patterns emitted by a fluorescent object under variable unknown random illumination, using matrix factorization and a fingerprint-based reconstruction.

(4) We proposed a multi-optical memory effect ranges imaging method through the scattering media using multi-frame speckle illumination. In this method, the speckle fingerprints of different point sources are achieved by using a demixing algorithm, and then the speckle fingerprints are classified by the speckle classification approach. Image reconstruction was performed separately for the optical memory effect range. At the same time, the application field of this method is not limited to the spatial optical memory effect, but also applies to the time-dimensional optical memory effect and the spectral-dimensional optical memory effect.
\\
\englishkeywords{scattering imgaging,\space{}speckle pattern,\space{}optical memory effect,\space{}non-invasive imaging,\space{}scattering media} \\

\end{englishabstract}


\XDUpremainmatter

\begin{symbollist}
\item [符号] \hspace{12em} {符号名称}
\item [$\mu_{a}$]\hspace{12.5em} {吸收系数}
\item [$\mu_{s}$]\hspace{12.5em} {散射系数}
\item [$l_{s}$] \hspace{12.5em} {平均自由程}
\item [$\langle \cdot \rangle$] \hspace{12.5em} {均值}
\item [$\bigstar$] \hspace{12.5em} {相关}
\item [$*$] \hspace{12.5em} {卷积}
\item [$\delta$] \hspace{12.5em} {$\delta$函数}
\item [$\mathcal{F}$] \hspace{12.5em} {傅里叶变换}
\item [$\mid \cdot \mid $] \hspace{12.5em} {模值}
\item [$\lambda$] \hspace{12.5em} {波长}
\item [$\mu$] \hspace{12.5em} {标准差}
\item [$(\cdot)^{(3)}$] \hspace{12.5em} {三阶相关}
\item [$\mathbb{R} $] \hspace{12.5em} {实数}
\item [$\Arrowvert \cdot \Arrowvert_F^{2}$] \hspace{12.5em} {Frobenius范数}
\item [$\vert\vert \mathbf{f}\vert\vert_2  $] \hspace{12.5em} {$L_{2}$向量范数}
\item [$\vert\vert \mathbf{f} \vert\vert_{TV} $] \hspace{12.5em} {TV范数}
\end{symbollist}

\begin{abbreviationlist}
\item \makebox[8em][l]{缩略语}  \makebox[16em][l]{英文全称}  \makebox[16em][l]{中文对照}
\item \makebox[4em][l]{OME}    \makebox[20em][l]{Optical Memory Effect}    \makebox[16em][l]{光学记忆效应}
\item \makebox[4em][l]{SLM}    \makebox[20em][l]{Spatial light modulator}    \makebox[16em][l]{空间光调制器}
\item \makebox[4em][l]{DMD}    \makebox[20em][l]{Digital micromirror device}  \makebox[16em][l]{数字微镜器件}
\item \makebox[4em][l]{2D}    \makebox[20em][l]{Two Dimensional}    \makebox[16em][l]{二维}
\item \makebox[4em][l]{3D}   \makebox[20em][l]{Three Dimensional}    \makebox[16em][l]{三维}
\item \makebox[4em][l]{MEMS}   \makebox[20em][l]{Micro electro mechanical system}    \makebox[16em][l]{微机电系统调制器}
\item \makebox[4em][l]{MFP}    \makebox[20em][l]{Mean-Free Path}    \makebox[16em][l]{平均自由程}.
\item \makebox[4em][l]{SSTM}    \makebox[20em][l]{Spatial-Spectral Transmission Matrix}    \makebox[16em][l]{空间-光谱传输矩阵}
\item \makebox[4em][l]{SVD}    \makebox[20em][l]{Singular Value Decomposition}    \makebox[16em][l]{奇异值分解}.
\item \makebox[4em][l]{HIO}    \makebox[20em][l]{Hybrid Input-Output}    \makebox[16em][l]{混合输入输出}
\item \makebox[4em][l]{ER}    \makebox[20em][l]{Error reduction}    \makebox[16em][l]{误差减小}
\item \makebox[4em][l]{TR}    \makebox[20em][l]{Tikhonov regularization}    \makebox[16em][l]{吉洪诺夫正则化算法}
\item \makebox[4em][l]{CVX}   \makebox[20em][l]{Convex Optimization}    \makebox[10em][l]{凸优化算法}
\item \makebox[4em][l]{FWHM}    \makebox[20em][l]{Full Width Half Maximum}    \makebox[16em][l]{全宽半高}
\item \makebox[4em][l]{sPSF}    \makebox[20em][l]{Spectral Point Spread Function}    \makebox[16em][l]{光谱点扩散函数}
\item \makebox[4em][l]{OTF}    \makebox[20em][l]{Optical Transfer Function}    \makebox[16em][l]{光学传递函数}
\item \makebox[4em][l]{prGAMP}    \makebox[20em][l]{Phase Retrieval via Generalized}    \makebox[16em][l]{广义近似信息传递相位恢复}
\item \makebox[4em][l]{\quad{}}    \makebox[20em][l]{ Approximate Message Passing}    \makebox[16em][l]{\quad{}}
\item \makebox[4em][l]{TV}    \makebox[20em][l]{Total variation}    \makebox[16em][l]{总变差}
\item \makebox[4em][l]{ALS}    \makebox[20em][l]{Alternating Least Squares}    \makebox[16em][l]{交替最小二乘法}
\item \makebox[4em][l]{FBR}    \makebox[20em][l]{Fingerprint-based reconstruction}    \makebox[16em][l]{基于散斑指纹的图像重建}
\item \makebox[4em][l]{SSIM}    \makebox[20em][l]{Structural Similarity Index Metric}    \makebox[16em][l]{结构相似性指数度量}
\item \makebox[4em][l]{MDS}    \makebox[20em][l]{Multidimensional scaling}    \makebox[16em][l]{多维缩放}
\end{abbreviationlist}
