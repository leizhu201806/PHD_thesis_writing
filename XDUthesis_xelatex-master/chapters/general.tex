
\chapter{绪论}\label{chap:1}
\section{研究背景和意义}
光学成像(人眼与大脑联合工作)可能是有史以来第一种用于科学研究的方法。虽然眼睛是奇妙的进化产物,但它有很大的局限性。我们的眼睛仅限于检测电磁波谱段一个非常小的区域,即“可见光区域”,其波长范围大约为 $400 nm$ 到 $800 nm$。我们的眼睛在远距离观测或极小物体分辨方面的能力有着极大的限制。为了克服这些问题,我们分别开发了望远镜和显微镜们。关于第一台显微镜(由两个或更多透镜组成,放大倍率$ \times 3 \sim 4$)的发明仍然存在一些争议,通常认为在1595年左右由荷兰Zacharias Janssen(1580-1638)所发明。随后,Robert Hooke(1635-1703)发现了许多事物中的弹性定律(即,胡克定律),进一步改进了显微镜(放大倍率高达 $ \times 30$,分辨率约为$5 um$)。特别是,他设计了一种新型螺旋式聚焦结构,在显微镜旁边放置了一个“水透镜”,将油灯发出的光聚焦在需要观测标本上提供照明。1665年,Robert Hooke出版了《Micrographia》\cite{robert_micrographia_2015}一书,这是一本具有历史意义的著作,他在讨论软木结构时创造了“细胞”一词,也在描述苍蝇、羽毛和雪花时提供详细的插图。早期的显微镜通常受到光学系统像差的限制,它会导致图像模糊和分辨率降低。直到19世纪后期,德国科学家 Carl Zeiss(1816-1888)开始研究如何改进显微镜成像系统,光学系统像差的限制才逐步被克服。当时,生产高质量的光学器件是通过反复试验进行的。为了建立更可靠的评价方法,Carl Zeiss 邀请德国数学家物理学家 Ernst Abbe一起进行了探索研究,最终他们提出了计算光学显微镜分辨率的原始公式\cite{abbe_beitrage_1873}。

得益于先进制造技术和光学成像方法的不断发展,光学成像在科学研究、生物学和生命科学等领域中展现出了独特的优势。得益于分子的特异性,线性荧光显微成像技术在生物学和生物医学科学等方面扮演了重要的角色。随着荧光显微成像技术的进一步发展,荧光显微成像技术不仅能够供结构信息和功能信息\cite{Lichtman2005},且具有高分辨率、高对比度和高速等特点。伴随着现代光学技术的发展,共聚焦显微镜或多光子显微镜不仅能够提供高对比的图像,而且在实现三维成像方面表现出了天然的优势。然而,当研究中样品的厚度增加时,只有其表面部分可以被完整成像。当光穿透到更深厚度的样品时,由于介质折射率不均匀引起光的多重散射现象。多重散射情况下,弹道光随深度增加呈指数衰减,不同光之间干涉形成散斑\cite{Goodman1976}。受到散射效应的影响,所有传统的显微技术在超出了传输平均自由程(在生物组织中通常为 $\simeq 1 mm$ \cite{ntziachristos_going_2010})后都迅速失效。

但是,光散射尽管非常复杂,却是确定的和可重复的,相同出射波产生固定的散斑图案。理论上,表征该过程和/或控制出射场应该是可能的,只要提供足够的自由度来调制入射场便可以控制散射效应。最值得注意的是,空间光调制器问世后,波前整形技术也应运而生,该技术具有在强烈无序的材料中操纵相干光的能力。特别是,它可用于消除散射效应(即,时间反演)并将光聚焦到衍射极限点\cite{Vellekoop2007}。该工作\cite{Vellekoop2007}通过测量目标焦点的信号,并将其作为反馈信号对入射波前进行校正。由于大多数显微镜技术总是涉及某种聚焦,这项开创性工作以及相关工作的贡献为更高分辨率光学成像带来了新的视角。尽管该方法的问世意味我们带克服光学散射方面迈进了一大步,但以非侵入方式测量来散射介质内的反馈信号更具挑战性。为了更适合生物医学成像应用,最近的诸多工作提出了利用导星(Guide star)机制实现非入侵反馈信号测量。该方法与波前调制技术具有相同功能,均能够实现透过散射介质的聚焦和成像。与此同时,散斑相关成像技术也应用而生,该技术基于光学记忆效应实现了透过散射介质非入侵成像。在下面小节,我们将对国内外散射成像研究现状进行概述。

\section{国内外研究现状}

光学相位共轭(Optical Phase Conjugation)是时间反演技术在散射方面的应用\cite{derode_robust_1995,draeger_one_channel_1997,leith_holographic_1966,fink_acoustic_2001},其核心是通过记录光场信息并生成相位共轭光的方式,实现入射光场的重现。其主要的步骤分为两步:(\romannum{1})记录输出光场的信息,利用通过添加参考光束的方式,实现干涉,并恢复光场信息;(\romannum{2})记录的光场信息展示在特定光学器件上,光场被反向传播,透过散射介质,形成聚焦点。随着空间光调制器 (Spatial light modulator,SLM)和数字微镜器件 (Digital micromirror device,DMD)的光学元件的发展,数字光学相位共轭(Digital Optical Phase Conjugation)变得流行,实现了透过散射介质聚焦\cite{yaqoob_optical_2008,paurisse_phase_2009,cui_implementation_2010,lhermite_coherent_2010}。重要的是,探测器和SLM位于分束器的镜面共轭平面中,它们必须逐个像素完美匹配,这使得实验对准变得非常繁琐且复杂。数字光学相位共轭被用来实现透过散射介质的聚焦,不同偏振态聚焦的控制等。对于数字光学相位共轭来说,该技术的缺点在于需要繁琐的光学系统校准;它的优点在于不需要更多的反馈优化步骤,只需两步便可实现聚焦。

与此同时,波前调制技术(Wave-Front Shaping)\cite{Vellekoop2007}也应运而生,其核心思想为利用线性或者非线性的优化算法,将输出光场作为反馈信号,对入射光场进行调制,进而获得特定的输出光场。2007年I. M. Vellekoop等人\cite{Vellekoop2007}首先提出了波前整形技术,并首次进行了实验验证,实现了利用透过散射介质的散射光聚焦。本质上,基于反馈优化的波前整形技术将散射介质对于光场的调制过程看作“黑箱”处理,通过迭代算法获取相应的波前,进而实现了对于输出光场的模式以及不同模式之间耦合的控制。与此同时,许多科研成果证明,在传统光学系统中加入散射介质,能够有效增加光学系统的分辨率\cite{vellekoop_exploiting_2010,choi_overcoming_2011}。其本质在于一定程度上扩大了光学系统的数值孔径(Numerical Aperture),增加了光学系统内的高频信息,实现光学分辨率的增加。随着波前整形技术的问世,如何有效的利用散射现象,控制散射特性成为了热点课题。对于波前整形来说,该技术的缺点在于:优化是一个迭代过程,因此相对较慢。此外,要控制任何其他空间不相关的输出位置,必须从头开始重新优化。同时它具有以下优点:迭代优化算法实现起来非常简单,它只需要探测器和SLM之间的闭环;它的简单性推动了在其它复杂光学系统的应用,例如多模光纤;同时它也是一种灵活的技术,因为用户可以最大化的自定义算法并且可以很容易地变更。同时,波前整形技术不仅能够实现单点聚焦,也能同时实现多点或多目标聚焦优化。针对不同的应用场景,选择合适的优化反馈函数,能够极大限度的挖掘系统信息,能够实现不同的目的聚焦或成像,例如利用荧光信息作为反馈信息\cite{boniface_non_invasive_2019}。

前面介绍的两种方法(数字光学相位共轭和波前整形)都依赖于光波传播和弹性散射的共同特性:即光传播方程的线性和时不变性。值得注意的是,这些特性在介质内的任何深度都有效,即使在多重散射状态下也是如此。这意味着无论多么复杂的介质,都可以通过线性变换来描述。在介观物理学中,这个问题是众所周知的,这种线性变换由散射矩阵$\mathbf{S}$描述,它将所有输入模式和所有输出模式建立连接,即特定输入和特定区域输出的结构。在成像环境中考虑到复杂结构时,通常考虑平面几何更加方便,介质有两个面,光照射在介质一侧并在另一侧收集的光。在这种平面几何问题中,将一组输入模式连接到一组输出模式的形式叫做传输矩阵(Transmission Matrix),并且所测量的矩阵是完整系统矩阵的子部分。2010年Popoff等人\cite{Popoff2010}首先利用四步相移干涉法测得了ZnO散射介质的光学传输矩阵。在完成测量传输矩阵工作后,可以实现聚焦或成像相关输出光场控制。对于传输矩阵技术来说,它的缺点在于:传输矩阵技术作为数字光学相位共轭技术的一部分,传输矩阵需要测量输出光场,需要通过增加的参考光(内部或外部)来完成矩阵测量;它的优点在于:与波前整形或数字光学相位共轭相比,传输矩阵的一个优点是它的测量包含输入场和整组输出像素之间的连接,其包含了更多的信息(远远超过波前整形或数字光学相位共轭可以获取的信息)。当完成传输矩阵测量后,它可以用于不同的目的:不仅可以实现不同点的聚焦,还可以通过计算的方式来研究介质的特征通道。到目前为止,我们描述了SLM 如何通过不同的方式来“消除”散射效应。这种将光聚焦在不透明材料后面的能力可以找到许多应用,例如生物学应用,它可以通过使用光遗传学工具(例如天然光敏离子通道或钙和电压敏感蛋白)来监测和控制特定群体细胞(例如神经元)的活动\cite{petreanu_channelrhodopsin-2assisted_2007}。

然而,为了实现成像,聚焦是不够的,通常需要一个额外的步骤,需要扫描整个样本的焦点并记录它发出的荧光,该工作通常在点扫描显微镜中完成。例如,2010年,I. M. Vellekoop等人\cite{vellekoop_scattered_2010}通过波前整形的方式透过散射介质实现聚焦,并通过平移聚焦点的方式,实现了透过散射介质的扫描成像。2012年,O. Katz等人\cite{katz_looking_2012}意识到透过散射介质透过散射介质进行扫描成像并不是唯一的成像方式,他们的设想是将散射介质变成“透镜”。他们将SLM放置在散射介质后面距离$R$处,然后对点源的散射光的相位进行整形,使其在相机特定点进行重新干涉聚焦。因此,SLM有效地将散射介质换为焦距为$R$的“透镜”。本质上,他们进行了聚焦实验的模拟,只是SLM放置在介质之后而不是之前,SLM与散射介质的输出平面共轭,以使记忆角范围最大化。由于光学记忆效应的限制,单波前校正对附近点有效,任何放置在初始点源位置附近的物体都会直接在相机上实时成像。为了证明这一点,他们将点源替换为扩展源,并在相机上直接获得图像。正如我们之前所见,深度扫描焦点可以直接从传输矩阵完成,但波前整形和数字光学相位共轭方法需要额外的测量。然而,由于散射介质的固有散斑相关性或其记忆效应,通过利用散斑相关性,能够极大的简化该成像过程,但是此过程往往难度较大,且引入大量的计算工作。

随着随散射研究工作的深入,2012年J. Bertolotti等人\cite{bertolotti_non-invasive_2012}基于光学记忆效应实现了非入侵光学散斑自相关成像,该方法无需侵入样本,完全以非入侵的形式实现了透过散射介质成像,该工作被Nature杂志评为2012年TOP10突破性方法。这种方法利用了一个基本理论,尽管散斑是一种非常复杂的模式,但它们具有明确的自相关峰值,即光学记忆效应(Optical Memory Effect, OME)\cite{Freund1988},通过计算散斑自相关恢复了隐藏目标的傅里叶振幅信息,并结合相位恢复算法,实现了无任何先验知识的散射成像。但是该方法需要利用入射光进行多角度扫描,获得多角度入射光照明对应散斑,最终进行图像重建。

随后,2014年O. Katz等人\cite{katz_non-invasive_2014}基于J. Bertolotti所提出的散斑自相关成像方法,同时受到天文观测中散斑干涉测量方法的启发,提出了利用单帧图像实现透过散射介质成像方法,克服了照明光源角度扫描的问题。在此工作中,他们用空间非相干光照射透射板(作为物体),并直接根据它在散射层后面产生的散斑的自相关来重建它。在这些方法中,不需要校准(即无需使用SLM进行波前整形),因为散射层的确切散射特性并不重要。然而,这些技术在使用薄散射材料而不是体积样品时效果更好,以确保光透射和且拥有非常显着的记忆效应。

需要强调的是:虽然这两种方法\cite{bertolotti_non-invasive_2012,katz_non-invasive_2014}能够对隐藏的物体进行成像,但是它们都具有有限的视野(由OME决定),而且随着穿透深度的增加而减小。此外,基于自相关的技术是非侵入的,因为它不需要物理访问目标平面。因此在生物成像应用方面,它具有重要意义。虽然基于OME光学成像已经发展数十年,具有良好的实时性和非入侵特性,但是目前仍有多方面工作仍然值得深入研究:

(1)针对目前已有的相位散斑相关成像方法,琳琅满目的相位恢复算法拥有各自适用的场景和前提,从目前以有的研究成果来看,相位恢复算法往往需要通过多次随机的初始值尝试,直至获得满意的重建结果。而且,所恢复的结果往往丢失了原始隐藏目标的方向信息。这是由于所有的这种基于傅里叶振幅结合相位恢复算法恢复傅里叶相位的模式导致隐藏目标的方向信息丢失,通过自相关的方式计算获得傅里叶振幅的过程,傅里叶振幅信息几何中心对称,已经丢失了隐藏目标的方向信息,进而利用相位恢复算法基于傅里叶振幅信息的图像恢复策略理论上无法恢复目标的方向信息。当该种策略无法保证恢复隐藏目标的方向信息,这也导致所恢复的图像难以再次利用,例如:彩色成像,光谱成像等。利用合适的相位恢复策略,进行隐藏目标的图像恢复,确保隐藏目标的方向信息,将有助于该方法在更多方面的应用;

(2)现有的散斑相关成像方法基于一项基本原则,OME范围内的散斑之间的相关函数可以近似为$\delta$函数,成像范围受到OME的限制。当隐藏目标的尺寸大于当前OME范围时,如何实现隐藏目标的图像重建,显然散斑自相关结合相位恢复的方式难以实现。如何深入的挖掘散斑信息,实现透过散射介质的超光学记忆效应范围的成像将对生物成像即相关领域有着重要的意义。同时,如何实现完全非入侵的超光学记忆效应范围成像也值得我们思考并进行深入研究;

(3)当成像系统中拥有多个目标时,且不同的目标位于不同的光学记忆效应范围时,如何实现成像?此时,我们所获得的散斑为不同目标的非相干叠加,或者可以视为不同点光源对应的散斑的非相干叠加。由于不同的点光源所对应的散斑携带了散射介质光学记忆效应范围的信息,如何利用这些消息,实现透过散射介质非入侵多目标或多光学记忆效应范围成像更具挑战。显然,此项任务比问题(2)更具挑战性,往往会在实际生物医学成像问题中遇到,解决此项问题将对于生物医学成像有着巨大的意义。

在本小节中,我们只对目前已有的散射成像方法进行总体性论述,未对各种的透过散射成像方式的相关成果进行一一列举对比,此项工作将在第\ref{chap:2}章中进行。

\section{本文结构与创新点}

\subsection{论文主要内容和章节安排}
针对上述的散射成像方法以及现存的问题,本文的具体研究内容如下:

(1)深入研究散斑自相关成像原理和成像模型,对现有的散射成像光路进行研究,对散斑相关成像策略进行数字模拟复现和实验复现。分析当照明光源光谱带宽不同时对散斑相关成像重建图像质量的影响,从数学模型分析成像质量变化的核心原因;

(2)研究散射介质光学散射特性,分析不同波长入射光对于相同散射介质的散射特性差异。对基于强度光谱传输矩阵的基本工作进行数字和实验复现,分析不同算法对于光谱信号重建的影响。同时对光谱传输矩阵进行改造,进而利用改造后的光谱传输矩阵实现成像;

(3)研究散斑相关成像基本模型,分析散斑成像方法丢失隐藏目标方向信息的内在原因。在解决散斑相关成像目标方向信息丢失问题的基础上,尝试解决透过散射介质彩色成像问题,实现透过散射彩色成像。在解决散射彩色成像问题的前提下,尝试有机的结合不同类型的相位恢复算法,在确保恢复隐藏目标方向信息的前提下,提高重建图像的质量;

(4)深入研究散射介质的光学记忆效应范围的内核,提出了一种基于时变随机照明的透过散射介质的超光学记忆效应范围非入侵成像方法。该方法利用去混叠算法对多帧散斑进行去混叠,获得不同点光源的散斑指纹,然后通过成对去卷积的方式实现了透过散射介质的超光学记忆效应图像重建。

(5)针对散射成像时遇到多目标或多光学记忆效应范围难题,深入研究不同光学记忆效应范围内在联系,分析光学记忆效应范围内的散斑特性,提出透过散射介质非入侵成像多目标或多光学记忆效应范围成像方法,解决目前散射成像时遇到多目标或多光学记忆效应范围难题。

结合以上部分的研究内容,本文的章节安排具体如下,全文共分为七章,每章的研究工作具体如下:

第一章 \ 绪论。本章首先介绍了可见光成像的发展,以及光学成像的简单发展史,并对目前成像遇到的问题进行描述,描述了光学成像中遇到的散射现象,并由此引出了散射成像;其次对目前国内外散射成像方法进行概述,对光学相位共轭、波前整形和散斑计算成像的发展现状进行了陈述,并对各项技术的优缺点进行了简单总结,同时也引出了本文研究工作将要聚焦的研究工作,即:透过散射介质非入侵成像,并分析了该方法目前的所遇到的问题即关键问题。最后,列举了本文具体的研究内容、章节安排和本文特色与创新点;

第二章 \ 超弹道光范围光学成像方法。本章首先对光散射的基本概念进行介绍、散射的产生的机理和不同散射区域进行介绍。其次,将现有的散射成像方法分为两类进行介绍,即:波前整形和基于光学记忆效应的散射成像技术。波前整形技术部分,其包含光学相位共轭、基于反馈优化的波前整形和光学传输矩阵技术,该技术主要研究光波在介质中的传播规律以及特性,为散射效应的利用奠定基础。基于光学记忆效应的散射成像技术部分,包含散斑相关成像技术和点扩散函数工程成像技术两部分,该技术核心在于对于散斑信息的利用,利用的散斑分布特点实现透过散射介质成像及相关工作。最后,对目前现存的方法的优缺点和目前所遇到的问题进行总体描述,并再次强调了本文后续的工作主要通过计算的方式挖掘散斑信息,实现散射成像的相关工作;

第三章 \ 透过散射介质的光谱信息恢复和空间信息恢复。本章首先对基于光谱传输矩阵的光谱重建模型和散斑相关成像的模型进行了分别介绍,并对两种方法分别进行了各自的仿真验证。其次,对基于光谱传输矩阵的光谱重建方法和散斑相关成像方法进行了有机的结合,实现了隐藏目标的光谱信息恢复和空间信息恢复的数字和实验验证。然后,分析了不同照明光源带宽所引起的重建图像质量变化的原因。最后,受到了光谱传输矩阵的光谱多样性启发,对光谱传输矩阵方法进行扩展,并实现了利用光谱传输矩阵的透过散射介质成像的相关实验验证和分析。总体来说,本章的所涉及的理论研究和实验验证均为本论文后续的相关工作奠定了坚实的基础;

第四章 \ 基于三阶相关相位恢复的彩色成像方法。本章首先对基于三阶相关相位恢复算法的透过散射介质彩色成像方法的基本原理进行介绍,并描述了三阶相关相位恢复的基本数学理论,对采用该相位恢复算法散射成像进行仿真验证。其次,对基于三阶相关相位恢复的彩色成像方法进行实验验证,分别对于简单目标和复杂目标进行实验验证;但是对于复杂目标进行彩色成像时,需要借助于参考目标实现,并对借助于参考目标实现复杂目标彩色成像进行了实验验证;同时,也分析了基于三阶相位恢复算法的抗噪性能和目标方向保持等特性进行分析和讨论。最后,描述了如何将三阶相位恢复算法和相位恢复算法进行有机的结合,在保持目标方向信息的同时,改善了重建图像的质量;

第五章\  基于时变随机照明的透过散射介质超光学记忆效应范围非入侵成像。本章首先对基于时变随机照明的透过散射介质超光学记忆效应范围成像原理进行了介绍,其中包含了两部分:散斑指纹的去混叠和基于散斑指纹的图像重建;然后,对该成像方法进行了实验验证,分别对稀疏二维(Two Dimensional, 2D)荧光目标和复杂连续(Three Dimensional, 3D)荧光目标进行成像实验;最后,对该方法所设计的估计系统的秩、光学记忆效应范围、基于散斑指纹图像重建、不同方法的对比、散斑数量对于重建结果的影响和该方法的可扩展性进行讨论。

第六章\ 基于多帧散斑照明的散射介质多光学记忆效应范围成像。本章首先对多光学记忆效应范围散射成像模型进行了描述,并探索了不同光学记忆效应范围内的散斑的内在联系,如:空间域和频域。其次,提出了基于散斑分类的透过散射介质多光学记忆效应范围成像策略,并对不同的散斑分类方法进行仿真验证;最后;对基于散斑分类的透过散射介质多OME范围成像进行仿真验证。

第七章\ 总结与展望。对本文所做的研究工作进行概括及总结,并对后期的研究方向及工作重点进行展望。

\subsection{本文特色与创新点}

本文所作的工作以散斑相关成像模型为基础,分别从物理层面和数学模型方面深入挖掘了散斑所携带的信息,实现了透过散射介质的彩色成像、大视场非入侵成像和多OME范围成像,为散射成像在生物医学领域的进一步实际应用起到了一定的作用。具体创新点如下:

(1)首先从理论出发,分析并研究了基于光谱传输矩阵的光谱重建方法和散斑自相关成像技术,分别对它们进行了验证;通过数字仿真和实验验证的方式证明了基于光谱传输矩阵的光谱重建方法和散斑自相关成像技术的有效性,即:窄谱光源照明和宽谱光源照明的适用性;将以上两种方法进行有机的组合,实现了利用散射介质对目标光谱信息和空间信息的获取;其次,基于光谱传输矩阵的基本原理,将其扩展至空间域,利用光谱传输矩阵实现成像;

(2)提出了一种基于三阶相关相位恢复的彩色成像方法。该项共工作的核心在于:三阶相关相位恢复和振幅恢复步骤的相互独立性,实现了目标方向信息的确定性恢复。首次实验验证了无需波前整形,且非侵入的形式实现透过散射介质的彩色成像。相较于传统的透过散射介质彩色成像方法,该方法无需对于目标的先验知识,无需介质波前整形技术,在时效性方面更具优势。同时,该方法能够与传统的光谱成像方法进行有效的结合,实现透过散射介质的光谱成像。但是需要注意的是,对于复杂目标,该方法需要借助参考目标实现不同彩色通道或者光谱通道目标的相对位置确定;

(3)提出了一种基于时变随机照明的透过散射介质超光学记忆效应范围成像方法。目前已有的透过散射介质成像方法,难以克服光学视场受限问题,并且在受限的视场下,图像重建质量受到相位恢复算法的影响,难以确定的且高质量的完成图像重建。通过分析不同光学记忆效应范围的内在联系,利用多帧随机照明的方式获得不同的散斑图像,并利用去混叠算法实现不同点源目标的散斑指纹恢复,利用散斑指纹之间的相关特性,实现了透过散射介质超光学记忆效应范围成像。在次工作中,我们首先提出了基于散斑指纹的图像重建概念。

(4)提出了一种基于多帧散斑照明的散射介质多光学记忆效应范围成像方法。在第五章所进行的工作中,我们实现了超光学记忆效应范围成像,该问题的前提在于不同的光学记忆效应范围之间有重叠。然而,在实际生物医学成像中经常遇到成像系统拥有多个独立光学记忆效应范围,此问题更具挑战且更贴近实际应用。我们提出了散斑分类方法和散斑指纹图像重建方法结合的工作模式,实现了透过散射介质的多光学记忆效应范围成像,且该方法在时间维和光谱维光学记忆效应具有应用潜力。
