
\chapter{绪论}
\section{研究背景和意义}
光学成像——通过强大的人眼—大脑组合进行——可能是有史以来第一种用于科学研究的方法。虽然眼睛是进化的奇妙产物,但它有很大的局限性。我们的眼睛仅限于检测电磁波谱的一个非常小的区域,我们称之为“可见区域”,由 $400 nm$ 到 $800 nm$之间的波长组成。此外,我们在解析远距离或非常小的物体的能力方面受到限制。为了克服这个问题,我们开发了专门的仪器,分别是望远镜和显微镜来帮助我们。第一台复合显微镜(由两个或更多透镜组成,放大倍率$ \times 3 \sim 4$)的发明围绕它存在一些争议,但通常归因于1595年左右在荷兰的 Zacharias Janssen(1580-1638)。后来,Robert Hooke(1635-1703)发现了许多事物中的弹性定律(即,胡克定律),进一步改进了显微镜(放大倍率高达 $ \times 30$,分辨率约为$5 um$)。特别是,他设计了一种新的螺旋式聚焦机构,并在显微镜旁边放置了一个水透镜,将油灯发出的光聚焦在他的标本上,从而照亮它们。 1665 年,胡克出版了《Micrographia》\cite{robert_micrographia_2015},这是一本关于他的观察的具有历史意义的著作,他在讨论软木结构时创造了“细胞”一词,但也描述并提供了苍蝇、羽毛或雪花的详细插图。早期的显微镜受到通常称为像差的显着限制,它会导致图像模糊和分辨率损失。直到19世纪后期德国机械师 Carl Zeiss(1816-1888)开始生产显微镜,这些限制才被克服。当时,生产高质量的光学器件是通过反复试验进行的。为了建立更可靠的方法,Carl Zeiss 聘请了数学家和物理学家 Ernst Abbe,他最终提出了计算光学显微镜可实现的最大分辨率的原始公式\cite{abbe_beitrage_1873}。

由于光学成像的先进技术、方法和仪器不断发展,现在对研究、生物工业和医学中的各种应用产生了极大的兴趣。在最广泛使用的技术中,具有分子特异性的荧光显微镜非常适合研究生物标本和机制。它构成了一种独特的策略,能够以亚细胞分辨率、高选择性和速度跨越生物学领域,提供结构和功能信息\cite{Lichtman2005}。结合现代改进,例如共焦或多光子显微镜,它还提供了产生高对比度体积图像的可能性。然而,当研究中的样品变得太厚时,只有其表面被正确成像。到达下层的光会经历由波长尺度折射率不均匀性引起的多重散射。在多重散射状态中,弹道光随深度呈指数衰减,相干光产生扩展散斑图案\cite{Goodman1976}。因此,所有传统的显微技术都迅速失效,超出了传输平均自由程(在生物组织中通常为 $\simeq 1 mm$ \cite{ntziachristos_going_2010})。

但是,光散射尽管非常复杂却是确定性和可重复的,相同出射波产生固定的散斑图案。理论上,表征该过程和/或控制出射场应该是可能的,只要提供足够的自由度来调制入射场以规避散射的影响。实验技术的出现——最值得注意的是,空间光调制器导致了波前整形的出现,即在强烈无序的材料中操纵相干光的能力。特别是,它可用于反转散射效应并将光聚焦到衍射极限点\cite{Vellekoop2007}。这是通过首先测量来自目标焦点的反馈信号然后校正入射波前来实现的。由于大多数显微镜技术总是涉及某种聚焦,这项开创性工作以及其他贡献为深度高分辨率光学成像带来了新的视角。尽管它们代表了克服光学散射的优雅方法,但以非侵入方式测量来自组织内的反馈信号实际上具有挑战性。为了更适合生物应用,最近的工作提出了使用指南星机制测量反馈的替代策略。与波前调制技术具有相同功能,能够实现透过散射介质成像方法被同时发展。其主要利用散斑中所包含的信息,通过计算的方式实现图像重建或者信息获取。在下面小节,我们将对国内外研究现状进行概述。

\section{国内外研究现状}

受到光学相位共轭(Optical Phase Conjugation,OPC)和时间反演技术在散射方面的应用\cite{derode_robust_1995,draeger_one_channel_1997,leith_holographic_1966,fink_acoustic_2001},其核心是通过记录光场信息并生成相位共轭光的方式,实现入射光场的重现。其主要的步骤分为两步:(\romannum{1})记录输出光场的信息,利用通过添加参考光束的方式,实现干涉,并恢复光场信息;(\romannum{2})记录的光场信息展示在特定光学器件上,光场被反向传播,透过散射介质,形成聚焦点。随着SLM和DMD的光学元件的发展,数字光学相位共轭(Digital Optical Phase Conjugation,DOPC)变得流行,实现了透过散射介质聚焦\cite{yaqoob_optical_2008,paurisse_phase_2009,cui_implementation_2010,lhermite_coherent_2010}。重要的是,探测器和SLM位于分束器的镜面共轭平面中,它们必须逐个像素完美匹配,这使得实验对准变得非常繁琐且复杂。DOPC开始被用来实现透过散射介质的聚焦,不同偏振态聚焦的控制等。对于DOPC来说,该技术的缺点在于需要繁琐的光学系统校准;它的有点在于不需要更多的反馈优化步骤,只需两步便可实现聚焦。

与此同时,波前调制技术(Wavefront Shaping Technique,WST)\cite{Vellekoop2007}也应运而生,其核心思想为利用线性或者非线性的优化算法,将输出光场作为反馈信号,对入射光场进行调制,进而获得特定的输出光场。本质上,基于反馈优化的波前整形技术将散射介质对于光场的调制过程看作“黑箱”处理,通过迭代算法获取相应的波前,进而实现了对于输出光场的模式以及不同模式之间耦合的控制。与此同时,许多科研成果证明,在传统光学系统中加入散射介质,能够有效调高光学系统的分辨率\cite{vellekoop_exploiting_2010,choi_overcoming_2011}。其本质在于一定程度上扩大了光学系统的数值孔径(Numerical Aperture,NA),增加了光学系统内的高频信息,实现光学分辨率的增加。随着WST技术的问世,如何有效的利用散射现象,控制散射特性成为了热点课题。对于WST来说,该技术的缺点在于:优化是一个迭代过程,因此相对较慢。此外,要关注任何其他空间不相关的输出位置,必须从头开始重新启动新程序。同时它具有够多的优点:迭代优化算法实现起来非常简单:它只需要检测器和SLM之间的闭环。 它的简单性导致其用于其他光学系统,例如多模光纤;同时它也是一种灵活的技术,因为算法最大化的度量是用户定义的并且可以很容易地更改。以前,我们只报道了单个相机像素的强度最大化,但他们也能够使用单个 SLM 同时将光聚焦在多个目标上,该技术也能同时实现多点或多目标聚焦优化。针对不同的应用场景,选择合适的优化策略,能够极大限度的挖掘系统信息,能够实现不同的目的,例如利用荧光信息作为反馈信息\cite{boniface_non_invasive_2019}。

前面介绍的两种方法(DOPC和WST)都依赖于波传播和弹性散射的共同特性:传播方程的线性和时不变性。值得注意的是,这些特性在介质内的任何深度都有效,即使在多重散射状态下也是如此。这意味着场通过电介质结构的传播,无论其复杂性和厚度如何,都可以通过线性变换来描述。在介观物理学中,这个问题是众所周知的,这种线性变换由散射矩阵$\mathbf{S}$描述,它将所有输入场模式连接到所有输出场模式,建立输入和特定区域输出的结构。在成像环境中考虑复杂结构时,通常考虑平板几何更方便,其中介质有两个面,其中一个考虑照射在介质一侧并在另一侧收集的光。在这个几何中,将一组输入模式连接到一组输出模式的量是传输矩阵$\mathbf{T}$(Transmission Matrix,TM),并且是系统完整矩阵$\mathbf{S}$的子部分。在完成测量TM工作后,可以实现成像聚焦相关输出光场控制。对于TM技术来说,它的缺点在于:TM技术作为DOPC技术的一部分,TM需要测量输出光场,需要通过使用增加了一些技术复杂性的参考(内部或外部)来完成的;它的优点在于:与WST或DOPC相比,TM的一个优点是它的测量包含输入场和整组输出像素之间的连接,这代表了很多信息(远远超过WST或DOPC可以检索的信息。当完成TM测量后,它就可以用于不同的目的:不仅可以实现不同点的聚焦,还可以通过计算的方式来研究特征通道。

到目前为止,我们只描述了 SLM 如何通过不同的策略来“消除”散射效应。因此,这种将光聚焦在不透明材料后面的能力可以找到许多应用。关于生物学应用,它可以通过使用光遗传学工具(例如天然光敏离子通道或钙和电压敏感蛋白)来监测和控制特定群体细胞(例如神经元)的活动。

  然而,为了成像目的,聚焦是不够的,需要一个额外的步骤。例如,需要扫描整个样本的焦点并记录它发出的荧光,这通常在点扫描显微镜中完成。正如我们之前所见,深度扫描焦点可以直接从传输矩阵完成,但WST和DOPC方法需要额外的测量。然而,由于散射介质的固有散斑相关性或其记忆效应,该方案可以显着简化。


\section{本文结构与创新点}

\subsection{论文主要内容和章节安排}

\subsection{本文特色与创新点}
