\chapter{总结和展望}\label{chap:7}
\section{全文总结}
本文在深入研究了现有的基于OME散射成像方法,针对目前散射成像存在的问题,如:散射彩色成像困难、视场受限和多目标视场混叠等问题,进行了相关的深入研究,在一定程度上解决了目前所遇到的问题。在无需借助波前整形的前提下,通过计算的方法实现了透过散射介质彩色成像,解决了散射成像视场受限问题。在解决散射成像视场受限的基础上,着眼解决实际生物成像中遇到的多目标问题(即:多OME范围问题),提出了基于散斑分类的多帧散斑照明的多OME范围成像方法,一定限度的解决了该问题,实现了散射多目标成像。值得强调的是:我们首次提出了一种基于波动随机照明的透过散射介质超OME范围成像方法,该方法极大了解决了散射成像视场受限问题。本文所完成的主要具体工作如下:

(1)首先从理论出发,分析并研究了基于光谱传输矩阵的光谱重建方法和散斑自相关成像技术,分别对它们进行了验证;通过数字仿真和实验验证的方式证明了基于光谱传输矩阵的光谱重建方法和散斑自相关成像技术的有效性,即:窄谱光源照明和宽谱光源照明的适用性;将以上两种方法进行有机的组合,实现了利用散射介质对目标光谱信息和空间信息的获取;其次,基于光谱传输矩阵的基本原理,将其扩展至空间域,利用光谱传输矩阵实现成像;

(2)针对透过散射介质难以实现彩色成像的问题,提出了一种基于三阶相关相位恢复的彩色成像方法。该项共工作的核心在于:三阶相关相位恢复和振幅恢复步骤的相互独立性,实现了目标方向信息的确定性恢复。首次实验验证了无需波前整形,且非侵入的形式实现透过散射介质的彩色成像。相较于传统的透过散射介质彩色成像方法,该方法无需对于目标的先验知识,无需介质波前整形技术,在时效性方面更具优势。同时,该方法能够与传统的光谱成像方法进行有效的结合,实现透过散射介质的光谱成像。但是需要注意的是,对于复杂目标,该方法需要借助参考目标实现不同彩色通道或者光谱通道目标的相对位置确定;

(3)提出了一种基于波动随机照明的透过散射介质超OME范围成像方法。目前已有的透过散射介质成像方法,难以克服光学视场受限问题,并且在受限的视场下,图像重建质量受到相位恢复算法的影响,难以确定的且高质量的完成图像重建。通过分析不同OME范围的内在联系,利用多帧随机照明的方式获得不同的散斑图像,并利用去混叠算法实现不同点源目标的散斑指纹恢复,利用散斑指纹之间的相关特性,实现了透过散射介质超OME范围成像。在次工作中,我们首先提出了基于散斑指纹的图像重建概念。

(4)提出了一种基于多帧散斑照明的散射介质多OME范围成像方法。在第五章所进行的工作中,我们解决了超OME范围成像,该问题的前提在于不同的OME范围之间有重叠。然而当不同的OME范围之间无重建或者拥有多个独立OME范围的情况在实际生物医学成像中经常遇到,针对该问题我们提出了利用散斑分类方法和散斑指纹图像重建方法像结合的方式,解决了多OME范围无重叠成像问题,实现了多目标成像,且对不同OME范围的目标分别同时重建。

\section{研究展望}

综合相关文献及报道来看,散射成像技术有以下重点、难点亟需突破:1)波前整形技术的实时性仍需进一步提高,在实际应用过程中,散射介质往往具有时变特性,难以实现透过动态散射介质成像,可采用一种快速且准确的光学传输矩阵测量方法,实现透过动态散射介质聚焦或成像;2)目前的大部分波前整形技术的能量利用率较低,可设计一种能量利用率高且便于操控的波前整形技术,在更高能量利用率且快速的前提下实现透过散射聚焦;3)目前已有的波前整形技术无法实现透过散射介质的实时观测,需要设计一种无需重建过程的透过散射介质成像方法,如通过光学补偿的方式实现透过散射介质的实时成像;4)目前基于散斑相关成像方法的成像范围受限于OME,拓展OME的范围是未来研究的趋势,如通过研究散射介质的物理特性,利用介质自身物理特性(开通道和闭通道)实现OME的拓展,从而实现透过散射介质大视场成像;5)所有的基于点扩展函数工程的透过散射介质成像技术都需要提前测量系统的点扩展函数,如何实现无须测量点扩展函数实现成像是未来的研究难点。

除了解决上述难点问题以外,透过散射介质成像未来较有意义的研究方向为:1)如何实现介质内的快速聚焦或成像,这将对生物医学观测或疾病检测具有重要意义;2)目前的部分研究结果表明,大气或云雾等自然界复杂介质具有散射特性,研究大气或云雾等复杂介质形成散斑的条件,实现透过自然界复杂介质成像对远距离探测和信息感知具有重要意义;3)目前大部分透过散射介质成像的方法只能实现单一物理量的探测,如何通过单帧散斑实现多物理量的探测,对目标信号的探测、识别和分辨具有重要意义;4)随着微纳加工技术的进一步发展,通过定制散射介质实现定制成像,将会对新型化、小型化和集成化的探测成像系统的发展具有深远意义;5)随着深度学习和人工智能技术的发展,可将人工智能技术引入到散射成像,通过少量先验知识或通过先验知识迁移的方式实现透过散射介质成像。如何有效的挖掘散斑包含的介质特性信息和目标的多样性所带来的散斑差异,将会有助于我们更好的利用散射现象,让散射成为我们可控的一种有效工具。
