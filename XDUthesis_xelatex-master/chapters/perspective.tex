\chapter{总结和展望}\label{chap:7}
\section{全文总结}
本文深入研究了现有的基于OME散射成像方法,针对目前散射成像存在的问题,如:散射彩色成像困难、视场受限和多目标视场混叠等问题,进行了相关的深入研究,在一定程度上解决了目前所遇到的问题。在无需借助波前整形的前提下,通过计算的方法实现了透过散射介质彩色成像,解决了散射成像视场受限问题。在解决散射成像视场受限的基础上,着眼解决实际生物成像中遇到的多目标问题(即:多OME范围问题),提出了基于散斑分类的多帧散斑照明的多OME范围成像方法,一定限度的解决了该问题,实现了散射多目标成像。值得强调的是:我们首次提出了基于波动随机照明的透过散射介质超OME范围非入侵成像方法,该方法极大了解决了散射成像视场受限问题。本文所完成的主要具体工作如下:

(1)首先从理论出发,分析并研究了基于光谱传输矩阵的光谱重建方法和散斑自相关成像技术,分别对它们进行了验证;通过数字仿真和实验验证的方式证明了基于光谱传输矩阵的光谱重建方法和散斑自相关成像技术的有效性;其次,将光谱传输矩阵方法扩展至空间域,实现透过散射介质成像;

(2)提出了一种基于三阶相关相位恢复的彩色成像方法。该项共工作的核心在于:三阶相关相位恢复和振幅恢复步骤的相互独立性,实现了透过散射介质彩色成像。相较于传统的透过散射介质彩色成像方法,该方法无需对于目标的先验知识,无需介质波前整形技术,在时效性方面更具优势。同时,该方法能够与传统的光谱成像方法进行有效的结合,实现透过散射介质的光谱成像。但是需要注意的是,对于复杂目标,该方法需要借助参考目标实现不同彩色通道或者光谱通道目标的相对位置确定;

(3)提出了一种基于波动随机照明的透过散射介质的超光学记忆效应范围非入侵成像方法。该方法利用去混叠算法对多帧散斑进行去混叠,获得不同点光源的散斑指纹,然后通过成对去卷积的方式实现了透过散射介质的超光学记忆效应图像重建。

(4)提出了一种基于多帧散斑照明的散射介质多光学记忆效应范围成像方法。该方法首先通过去混叠算法对多帧散斑进行去混叠获取不同点光源的散斑指纹,然后利用散斑分类方式对散斑指纹进行分类,最后利用成对去卷积的方式对多光学记忆效应范围分别进行图像重建。该方法具有在时间维光学记忆效应和光谱维光学记忆应用的潜力。


\section{研究展望}

综合相关文献及报道来看,散射成像技术有以下重点、难点亟需突破:1)波前整形技术的实时性仍需进一步提高,在实际应用过程中,散射介质往往具有时变特性,难以实现透过动态散射介质成像,可采用一种快速且准确的光学传输矩阵测量方法,实现透过动态散射介质聚焦或成像;2)目前的大部分波前整形技术的能量利用率较低,可设计一种能量利用率高且便于操控的波前整形技术,在更高能量利用率且快速的前提下实现透过散射聚焦;3)目前已有的波前整形技术无法实现透过散射介质的实时观测,需要设计一种无需重建过程的透过散射介质成像方法,如通过光学补偿的方式实现透过散射介质的实时成像;4)目前基于散斑相关成像方法的成像范围受限于OME,拓展OME的范围是未来研究的趋势,如通过研究散射介质的物理特性,利用介质自身物理特性(开通道和闭通道)实现OME的拓展,从而实现透过散射介质大视场成像;5)所有的基于点扩展函数工程的透过散射介质成像技术都需要提前测量系统的点扩展函数,如何实现无须测量点扩展函数实现成像是未来的研究难点。

除了解决上述难点问题以外,透过散射介质成像未来较有意义的研究方向为:1)如何实现介质内的快速聚焦或成像,这将对生物医学观测或疾病检测具有重要意义;2)目前的部分研究结果表明,大气或云雾等自然界复杂介质具有散射特性,研究大气或云雾等复杂介质形成散斑的条件,实现透过自然界复杂介质成像对远距离探测和信息感知具有重要意义;3)目前大部分透过散射介质成像的方法只能实现单一物理量的探测,如何通过单帧散斑实现多物理量的探测,对目标信号的探测、识别和分辨具有重要意义;4)随着微纳加工技术的进一步发展,通过定制散射介质实现定制成像,将会对新型化、小型化和集成化的探测成像系统的发展具有深远意义;5)随着深度学习和人工智能技术的发展,可将人工智能技术引入到散射成像,通过少量先验知识或通过先验知识迁移的方式实现透过散射介质成像。如何有效的挖掘散斑包含的介质特性信息和目标的多样性所带来的散斑差异,将会有助于我们更好的利用散射现象,让散射成为我们可控的一种有效工具。
