
\begin{thanks}

很高兴的开始书写致谢部分,在开始本章之前,我认为它可能是最简单且最容易的章节,但是此时此刻,我想说可能这是本博士论文对于我来说最难的时刻,甚至不知道如何开始本章的第一句。如果幸运的话,这里的一句话意味着博士生涯的即将结束,我希望我足够幸运。经过了六年的博士学习与生活,或许我该说六年的生活与学习,这个过程中教会我更多的是如何生活。在这个不长不短地博士生涯中,太多收获,收获了我自己地家庭,收获了人生中地第一篇文章,收获了第一次睁眼看世界的机会,收获第一次的单人国外旅行......颇丰的收获,让我只能用省略号结束上一句话,省略号的意义不是那些收获被省略,他们太重要已经印在了我的记忆中,难以忘记。许多让我感动的事,许多触动我的人,每一时刻的感动,每一点滴的生活,汇合成今天的河流,让我不得不感动。回首博士生涯,最重要的是来自身边人的教导、关心和帮助。你们的耐心和坚持让我在面对困难的时候选择了直面问题。鲁迅说过:真正的勇士,敢于直面惨淡的人生,敢于正视淋漓的鲜血。现在的我仍然不是勇士,依旧需要你们的帮助和呵护。在你们的羽翼下,才以自由的生长。我要向你们表示衷心的感谢!谢谢你们。

首先,我要感谢我的博士导师邵晓鹏教授。邵老师在生活上到学习上给予我太多帮助,让我能够顺利的完成博士学业。邵老师在科研方面的视野以及对于工作的态度,给了我更多的鼓励和启发。六年的时光里,邵老师教会了我学习和科研的方法,同时也鼓励出国学习,磨练自己,这份经历让我受益匪浅。邵老师做到了授人以鱼不如授人以渔。

其次,我要感谢我的法国导师Prof. Sylvain Gigan。感谢Prof. Sylvain Gigan提供了去法国学习工作的机会。在这两年多的岁月里,了解到了不同的学习氛围与学习方式,参与到更多的讨论与分享。不同的文化的熏陶,让我更明白如何去生活,如何去珍惜时间。特别是在我遇到困难时,每次的会面与交谈都铭记于心。你的支持让我能够面对问题,并解决问题。

感谢国家留学基金委提供的奖学金,使我获得了赴法留学的机会,培养了我独立生活与学习的能力,这对我的人生具有重要的意义。

感谢课题组的王学恩、龚睿、刘杰涛、刘飞、吴雨祥和各位老师,师兄师姐师弟师妹们,他们在科研工作和生活中给予了很多帮助。

感谢与我在法国实验室COMPLEX MEDIA OPTICS LAB相遇的各位,Dr. Hilton B. de Aguiar,Dr. Claudio Moretti,Dr. Bernhard Rauer,Dr. Lorenzo Valzania,Dr. Fernando Soldevila Torres,Dr. Gianni Jacucci,Dr. YoonSeok Baek,Dr. DABROWSKI Michal,Dr. Raj Pandya,Dr. Fei Xia,Antoine Boniface,Jonathan Dong,Louisiane Devaud,Julien Guilbert,Alexandra D’Arco,Lea Chibani,Louis Delloye。感谢你们的友好和照顾让我在法国感受到家人的温暖。特此感谢我的法语老师Louisiane Devaud,Je te dis que tu es la meilleure professeur de français.

感谢在法国遇到的各位同学们,以及所有帮助过我的人,是你们让我在法国没有感到孤独和无助。

最后,我要特别感谢我的家人们,你们的支持和理解让我的学习和生活变得更加有意义。特别是我的老婆田冰心,你的支持和理解让我坚持走了下来,

Light thinks it travels faster than anything but it is wrong. No matter how fast light travels, it finds the darkness has always got there first, and is waiting for it.-Terry Pratchett, Reaper Man.
\end{thanks}
