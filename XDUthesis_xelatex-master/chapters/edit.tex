
\chapter{基于三阶相关相位恢复的彩色成像方法}

通过散射介质彩色成像对于深层组织的非侵入性成像和其他生物医学用途非常重要,进一步的发展将有利于生物医学应用。 随着空间光调制器技术的发展,利用波前整形技术通过散射介质实现彩色成像已成为现实。 然而,波前整形技术非常耗时,因为需要很长的测量步骤序列,每个成像像素一个,因此很难在没有反馈的情况下使用。 Sahoo 等人最近的一项突破性方法。利用谱点扩展函数 (sPSF) 的去相关性,实现了多光谱

在前面章节中,我们对散斑相关成像的方法进行了阐述,并进行了相关实验验证,实验证明了基于光学记忆效应的散斑相关成像方法能够有效的实现对隐藏目标的成像。该方法的核心思想为:通过计算散斑的自相关,移除掉系统PSF的影响,根据维纳辛钦定律进而获得隐藏目标的傅里叶振幅信息。以恢复隐藏目标的傅里叶振幅信息为支撑,利用相位恢复算法进而实现了隐藏目标的的傅里叶相位信息猜测,实现了隐藏目标的成像。常见的相位恢复算法需要尝试多次的随机初始猜测,才能较好的恢复图像,但是仍然难以保证正确的恢复隐藏目标的方向信息。当所恢复的隐藏目标方向信息不能保证时,对于透过散射散射介质的彩色成像造成了更大困难。我们是否能够找到恰当的相位恢复算法,确定性的恢复目标,进而实现透过散射介质的彩色成像?

在本章中,我们提出了一种基于三阶相关相位恢复的透过散射介质的彩色成像方法。在这里,我们从理论上证明了三重相关技术的空间平均可用于检索物体的傅立叶相位,并且我们通过实验证明它可以通过散射介质应用于彩色成像。 与其他相位恢复索技术相比,三阶相关相位恢复技术可以保留物体的方位信息,无需旋转操作即可合成彩色图像。此外,我们的方法有可能通过散射介质实现光谱成像。

\section{现状}

学位论文的封面由研究生院按国家规定统一制定印刷,封面内容必须打印,不得手写。

\section{学位论文的版面设置要求}

(1)行间距:固定值~20~磅(题名页除外)。

(2)字符间距:标准。

(3)页眉设置:单面页码页眉标题为章节题目,每一章节的起始页必须在单面页码,双面页码页眉标题统一为“西安电子科技大学博/硕士学位论文”,页眉标题居中排列,字体为宋体,字号为五号。页眉文字下添加双横线,双横线宽度为~0.5~ 磅,距正文距离为:上下各~1~磅,左右各~4~磅。

(4)页码设置:学位论文的前置部分和主体部分分开设置页码,前置部分的页码用罗马数字标识,字体为~Times New Roman~,字号为小五号;主体部分的页码用阿拉伯数字标识,字体为宋体,字号为小五号。页码统一居于页面底端中部,不加任何修饰。

(5)页面设置:为了便于装订,要求每页纸的四周留有足够的空白边缘,其中页边距为上~3~厘米、下~2~厘米;内侧~2.5~厘米、外侧~2.5~厘米;装订线为~0.5~厘米;页眉~2~厘米,页脚~1.75~厘米。

\section{学位论文的打印、装订要求}

(1)打印:学位论文必须用~A4~纸页面排版,双面打印;

(2)装订:依次按照中文题名页、英文题名页、声明、摘要、插图索引、表格索引、符号对照表、缩略语对照表、目录、正文、附录(可选)、参考文献、致谢、作者简介的顺序,用学校统一印制的学位论文封面装订成册。盲审论文必须删除致谢部分的文字内容(致谢标题须保留)以及封面和研究成果中的作者和指导教师姓名,研究成果列表中应体现作者的排序,如第一作者、第一发明人等。

\section{其他说明}

本规定由研究生院负责解释,从申请~2015~年~9~月毕业和授位的研究生开始执行,其它有关规定同时废止。研究生毕业论文撰写要求参照学位论文撰写要求执行。
